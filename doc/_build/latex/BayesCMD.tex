%% Generated by Sphinx.
\def\sphinxdocclass{report}
\documentclass[letterpaper,10pt,english]{sphinxmanual}
\ifdefined\pdfpxdimen
   \let\sphinxpxdimen\pdfpxdimen\else\newdimen\sphinxpxdimen
\fi \sphinxpxdimen=.75bp\relax

\usepackage[utf8]{inputenc}
\ifdefined\DeclareUnicodeCharacter
 \ifdefined\DeclareUnicodeCharacterAsOptional
  \DeclareUnicodeCharacter{"00A0}{\nobreakspace}
  \DeclareUnicodeCharacter{"2500}{\sphinxunichar{2500}}
  \DeclareUnicodeCharacter{"2502}{\sphinxunichar{2502}}
  \DeclareUnicodeCharacter{"2514}{\sphinxunichar{2514}}
  \DeclareUnicodeCharacter{"251C}{\sphinxunichar{251C}}
  \DeclareUnicodeCharacter{"2572}{\textbackslash}
 \else
  \DeclareUnicodeCharacter{00A0}{\nobreakspace}
  \DeclareUnicodeCharacter{2500}{\sphinxunichar{2500}}
  \DeclareUnicodeCharacter{2502}{\sphinxunichar{2502}}
  \DeclareUnicodeCharacter{2514}{\sphinxunichar{2514}}
  \DeclareUnicodeCharacter{251C}{\sphinxunichar{251C}}
  \DeclareUnicodeCharacter{2572}{\textbackslash}
 \fi
\fi
\usepackage{cmap}
\usepackage[T1]{fontenc}
\usepackage{amsmath,amssymb,amstext}
\usepackage{babel}
\usepackage{times}
\usepackage[Bjarne]{fncychap}
\usepackage[dontkeepoldnames]{sphinx}

\usepackage{geometry}

% Include hyperref last.
\usepackage{hyperref}
% Fix anchor placement for figures with captions.
\usepackage{hypcap}% it must be loaded after hyperref.
% Set up styles of URL: it should be placed after hyperref.
\urlstyle{same}
\addto\captionsenglish{\renewcommand{\contentsname}{Contents:}}

\addto\captionsenglish{\renewcommand{\figurename}{Fig.}}
\addto\captionsenglish{\renewcommand{\tablename}{Table}}
\addto\captionsenglish{\renewcommand{\literalblockname}{Listing}}

\addto\captionsenglish{\renewcommand{\literalblockcontinuedname}{continued from previous page}}
\addto\captionsenglish{\renewcommand{\literalblockcontinuesname}{continues on next page}}

\addto\extrasenglish{\def\pageautorefname{page}}

\setcounter{tocdepth}{1}



\title{BayesCMD Documentation}
\date{Oct 12, 2017}
\release{}
\author{Joshua Russell-Buckland}
\newcommand{\sphinxlogo}{\vbox{}}
\renewcommand{\releasename}{Release}
\makeindex

\begin{document}

\maketitle
\sphinxtableofcontents
\phantomsection\label{\detokenize{index::doc}}


BayesCMD is a package intended to expand the capabilities of the
Brain/Circulation Modelling (BCMD) framework. It introduces the ability to
obtain posterior distributions for model parameters by using Approximate
Bayesian Computation (ABC).


\chapter{bcmdModel}
\label{\detokenize{bcmdModel::doc}}\label{\detokenize{bcmdModel:welcome-to-bayescmd-s-documentation}}\label{\detokenize{bcmdModel:bcmdmodel}}

\section{Running the BCMD Model}
\label{\detokenize{bcmdModel:running-the-bcmd-model}}
The BCMD model can be run in a number of ways, both using the command lines
and the WeBCMD interface. Over time, both the BayesCMD package and the WeBCMD
package are expected to merge. As a result, the BCMD model class has been
designed to allow flexibility and compatibility with both the current BayesCMD
framework and the WeBCMD framework.


\subsection{bcmd\_model}
\label{\detokenize{bcmdModel:bcmd-model}}\label{\detokenize{bcmdModel:module-bayescmd.bcmdModel.bcmd_model}}\index{bayescmd.bcmdModel.bcmd\_model (module)}
Configure and run a BCMD model.

The BCMD Model class is used to configure and run a BCMD model. It is done by
passing a number of configuration variables, creating a model input file and
then running the model. Models can also be run from a pre-existing input file
and input files can be written to file for later access also.
\index{TIMEOUT (in module bayescmd.bcmdModel.bcmd\_model)}

\begin{fulllineitems}
\phantomsection\label{\detokenize{bcmdModel:bayescmd.bcmdModel.bcmd_model.TIMEOUT}}\pysigline{\sphinxcode{bayescmd.bcmdModel.bcmd\_model.}\sphinxbfcode{TIMEOUT}}
\sphinxcode{int} \textendash{} Max number of seconds for a simulation to run before being cancelled.
Default is 30 seconds.

\end{fulllineitems}

\index{BASEDIR (in module bayescmd.bcmdModel.bcmd\_model)}

\begin{fulllineitems}
\phantomsection\label{\detokenize{bcmdModel:bayescmd.bcmdModel.bcmd_model.BASEDIR}}\pysigline{\sphinxcode{bayescmd.bcmdModel.bcmd\_model.}\sphinxbfcode{BASEDIR}}
\sphinxcode{str} \textendash{} Path to Base directory, which should be ‘BayesCMD’. It is found by running
the {\hyperref[\detokenize{misc:bayescmd.util.findBaseDir}]{\sphinxcrossref{\sphinxcode{bayescmd.util.findBaseDir}}}} method, passing either an environment
variable or a string to the method.

\end{fulllineitems}

\index{ModelBCMD (class in bayescmd.bcmdModel)}

\begin{fulllineitems}
\phantomsection\label{\detokenize{bcmdModel:bayescmd.bcmdModel.ModelBCMD}}\pysiglinewithargsret{\sphinxbfcode{class }\sphinxcode{bayescmd.bcmdModel.}\sphinxbfcode{ModelBCMD}}{\emph{model\_name}, \emph{inputs=None}, \emph{params=None}, \emph{times=None}, \emph{outputs=None}, \emph{burn\_in=999}, \emph{create\_input=True}, \emph{input\_file=None}, \emph{suppress=False}, \emph{workdir=None}, \emph{deleteWorkdir=False}, \emph{timeout=30}, \emph{basedir='..'}, \emph{debug=False}, \emph{testing=False}}{}
Use to create inputs, run simulations and parse outputs.
\begin{quote}\begin{description}
\item[{Parameters}] \leavevmode\begin{itemize}
\item {} 
\sphinxstyleliteralstrong{model\_name} (\sphinxcode{str}) \textendash{} Name of model. Should match the modeldef file for model being generated
i.e. model\_name of ‘model{}`’ should have a modeldef file
‘model1.modeldef’.

\item {} 
\sphinxstyleliteralstrong{inputs} (\sphinxcode{dict} or \sphinxcode{None}, optional) \textendash{} Dictionary of model inputs and their values. Has form
\{‘names’ : \sphinxcode{list} of \sphinxcode{str},
‘values’ : \sphinxcode{list} of \sphinxcode{list} of \sphinxcode{float}\}
where \sphinxtitleref{names} should be a list of each model input name, matching up to
the model inputs and \sphinxtitleref{values} would be a list of lists, where each
sublist is the input values for that time point. With this in mind,
the length of \sphinxtitleref{inputs{[}‘values’{]}{}`} should equal length of \sphinxtitleref{times}.
Default is None.

\item {} 
\sphinxstyleliteralstrong{params} (\sphinxcode{dict} of \sphinxcode{str}: \sphinxcode{float} or \sphinxcode{None}, optional) \textendash{} Dictionary of \{‘parameter’: param\_value\}. Default is None

\item {} 
\sphinxstyleliteralstrong{times} (\sphinxcode{list} of \sphinxcode{float} or \sphinxcode{int} or \sphinxcode{None}, optional) \textendash{} List of times at which measurement data has been collected and needs
to be simulated. Default is None.

\item {} 
\sphinxstyleliteralstrong{outputs} (\sphinxcode{list} of \sphinxcode{str} or \sphinxcode{None}, optional) \textendash{} List of model outputs to return. Default is None.

\item {} 
\sphinxstyleliteralstrong{burn\_in} (\sphinxcode{float} or \sphinxcode{int}, optional) \textendash{} Length of burn in period at start of the simulation. Default is 999

\item {} 
\sphinxstyleliteralstrong{create\_input} (\sphinxcode{boolean}, optional) \textendash{} Boolean indicator as to whether an input file needs creating. Default
is True.

\item {} 
\sphinxstyleliteralstrong{input\_file} (\sphinxcode{str} or \sphinxcode{None}, optional) \textendash{} Path to existing input file or to where one needs creating. Default is
None.

\item {} 
\sphinxstyleliteralstrong{suppress} (\sphinxcode{boolean}, optional) \textendash{} Indicates if console output should be suppressed during model runs.
This will prevent writing of both stderr and stdout. Default is False.

\item {} 
\sphinxstyleliteralstrong{workdir} (\sphinxcode{str} or \sphinxcode{None}, optional) \textendash{} Path to working directory if one exists. If not set, it will default
to a temporary directory in ‘tmp/’. If you wish to write input files
and similar to file, it is recommended that the working directory is
set manually by the user.

\item {} 
\sphinxstyleliteralstrong{deleteWorkdir} (\sphinxcode{boolean}, optional) \textendash{} Indicates if the working directory should be deleted after finishing.
Default is False.

\item {} 
\sphinxstyleliteralstrong{timeout} (\sphinxcode{float} or \sphinxcode{int}, optional) \textendash{} Maximum length in seconds to let model run before cancelling. Default
is {\hyperref[\detokenize{bcmdModel:bayescmd.bcmdModel.bcmd_model.TIMEOUT}]{\sphinxcrossref{\sphinxcode{TIMEOUT}}}}.

\item {} 
\sphinxstyleliteralstrong{basedir} (\sphinxcode{str}, optional) \textendash{} Path to base ‘BayesCMD’ directory. By default it is set to
{\hyperref[\detokenize{bcmdModel:bayescmd.bcmdModel.bcmd_model.BASEDIR}]{\sphinxcrossref{\sphinxcode{BASEDIR}}}}.

\item {} 
\sphinxstyleliteralstrong{debug} (\sphinxcode{boolean}, optional) \textendash{} Indicates if debugging information should be written to console.
Default is False.

\item {} 
\sphinxstyleliteralstrong{testing} (\sphinxcode{boolean}, optional) \textendash{} If True, appends ‘\_test’ to coarse and detailed model output. Useful
if you wish to test settings and want to avoid test results becoming
mixed in with real result files. Default is False.

\end{itemize}

\end{description}\end{quote}
\index{model\_name (bayescmd.bcmdModel.bcmd\_model.ModelBCMD attribute)}

\begin{fulllineitems}
\phantomsection\label{\detokenize{bcmdModel:bayescmd.bcmdModel.bcmd_model.ModelBCMD.model_name}}\pysigline{\sphinxbfcode{model\_name}}
\sphinxcode{str} \textendash{} Name of model. Should match the modeldef file for model being generated
i.e. model\_name of ‘model{}`’ should have a modeldef file
‘model1.modeldef’.

\end{fulllineitems}

\index{inputs (bayescmd.bcmdModel.bcmd\_model.ModelBCMD attribute)}

\begin{fulllineitems}
\phantomsection\label{\detokenize{bcmdModel:bayescmd.bcmdModel.bcmd_model.ModelBCMD.inputs}}\pysigline{\sphinxbfcode{inputs}}
\sphinxcode{dict} or \sphinxcode{None} \textendash{} Dictionary of model inputs and their values. Has form
\{‘names’ : \sphinxcode{list} of \sphinxcode{str},
‘values’ : \sphinxcode{list} of \sphinxcode{list} of \sphinxcode{float}\}
where \sphinxtitleref{names} should be a list of each model input name, matching up to
the model inputs and \sphinxtitleref{values} would be a list of lists, where each
sublist is the input values for that time point. With this in mind,
the length of \sphinxtitleref{inputs{[}‘values’{]}{}`} should equal length of \sphinxtitleref{times}.
Default is None.

\end{fulllineitems}

\index{params (bayescmd.bcmdModel.bcmd\_model.ModelBCMD attribute)}

\begin{fulllineitems}
\phantomsection\label{\detokenize{bcmdModel:bayescmd.bcmdModel.bcmd_model.ModelBCMD.params}}\pysigline{\sphinxbfcode{params}}
\sphinxcode{dict} of \sphinxcode{str}: \sphinxcode{float} or \sphinxcode{None} \textendash{} Dictionary of \{‘parameter’: param\_value\}. Default is None

\end{fulllineitems}

\index{times (bayescmd.bcmdModel.bcmd\_model.ModelBCMD attribute)}

\begin{fulllineitems}
\phantomsection\label{\detokenize{bcmdModel:bayescmd.bcmdModel.bcmd_model.ModelBCMD.times}}\pysigline{\sphinxbfcode{times}}
\sphinxcode{list} of \sphinxcode{float} or \sphinxcode{int} or \sphinxcode{None} \textendash{} List of times at which measurement data has been collected and needs
to be simulated. Default is None.

\end{fulllineitems}

\index{outputs (bayescmd.bcmdModel.bcmd\_model.ModelBCMD attribute)}

\begin{fulllineitems}
\phantomsection\label{\detokenize{bcmdModel:bayescmd.bcmdModel.bcmd_model.ModelBCMD.outputs}}\pysigline{\sphinxbfcode{outputs}}
\sphinxcode{list} of \sphinxcode{str} or \sphinxcode{None} \textendash{} List of model outputs to return. Default is None.

\end{fulllineitems}

\index{burn\_in (bayescmd.bcmdModel.bcmd\_model.ModelBCMD attribute)}

\begin{fulllineitems}
\phantomsection\label{\detokenize{bcmdModel:bayescmd.bcmdModel.bcmd_model.ModelBCMD.burn_in}}\pysigline{\sphinxbfcode{burn\_in}}
\sphinxcode{float} or \sphinxcode{int} \textendash{} Length of burn in period at start of the simulation. Default is 999

\end{fulllineitems}

\index{create\_input (bayescmd.bcmdModel.bcmd\_model.ModelBCMD attribute)}

\begin{fulllineitems}
\phantomsection\label{\detokenize{bcmdModel:bayescmd.bcmdModel.bcmd_model.ModelBCMD.create_input}}\pysigline{\sphinxbfcode{create\_input}}
\sphinxcode{boolean} \textendash{} Boolean indicator as to whether an input file needs creating. Default
is True.

\end{fulllineitems}

\index{input\_file (bayescmd.bcmdModel.bcmd\_model.ModelBCMD attribute)}

\begin{fulllineitems}
\phantomsection\label{\detokenize{bcmdModel:bayescmd.bcmdModel.bcmd_model.ModelBCMD.input_file}}\pysigline{\sphinxbfcode{input\_file}}
\sphinxcode{str} \textendash{} Path to existing input file or to where one needs creating. If
{\hyperref[\detokenize{bcmdModel:bayescmd.bcmdModel.bcmd_model.ModelBCMD.create_input}]{\sphinxcrossref{\sphinxcode{create\_input}}}} is True and no path is given, the input file will
be written to {\hyperref[\detokenize{bcmdModel:bayescmd.bcmdModel.bcmd_model.ModelBCMD.workdir}]{\sphinxcrossref{\sphinxcode{workdir}}}} as {\hyperref[\detokenize{bcmdModel:bayescmd.bcmdModel.bcmd_model.ModelBCMD.model_name}]{\sphinxcrossref{\sphinxcode{model\_name}}}}.input.

\end{fulllineitems}

\index{suppress (bayescmd.bcmdModel.bcmd\_model.ModelBCMD attribute)}

\begin{fulllineitems}
\phantomsection\label{\detokenize{bcmdModel:bayescmd.bcmdModel.bcmd_model.ModelBCMD.suppress}}\pysigline{\sphinxbfcode{suppress}}
\sphinxcode{boolean} \textendash{} Indicates if console output should be suppressed during model runs.
This will prevent writing of both stderr and stdout. Default is False.

\end{fulllineitems}

\index{DEVNULL (bayescmd.bcmdModel.bcmd\_model.ModelBCMD attribute)}

\begin{fulllineitems}
\phantomsection\label{\detokenize{bcmdModel:bayescmd.bcmdModel.bcmd_model.ModelBCMD.DEVNULL}}\pysigline{\sphinxbfcode{DEVNULL}}
\sphinxcode{\_io.BufferedWriter} \textendash{} If {\hyperref[\detokenize{bcmdModel:bayescmd.bcmdModel.bcmd_model.ModelBCMD.suppress}]{\sphinxcrossref{\sphinxcode{suppress}}}} is set to True, this will be an io buffer that
redirects stderr and stdout to the null device.

\end{fulllineitems}

\index{workdir (bayescmd.bcmdModel.bcmd\_model.ModelBCMD attribute)}

\begin{fulllineitems}
\phantomsection\label{\detokenize{bcmdModel:bayescmd.bcmdModel.bcmd_model.ModelBCMD.workdir}}\pysigline{\sphinxbfcode{workdir}}
\sphinxcode{str} or \sphinxcode{None} \textendash{} Path to working directory if one exists. If not set, it will default
to a temporary directory in ‘tmp/’. If you wish to write input files
and similar to file, it is recommended that the working directory is
set manually by the user.

If no working directory is given, the {\hyperref[\detokenize{bcmdModel:bayescmd.bcmdModel.bcmd_model.ModelBCMD.deleteWorkdir}]{\sphinxcrossref{\sphinxcode{deleteWorkdir}}}} attribute
will be set to True in order to ensure that the file space does not
become excessively full during batch runs.

\end{fulllineitems}

\index{deleteWorkdir (bayescmd.bcmdModel.bcmd\_model.ModelBCMD attribute)}

\begin{fulllineitems}
\phantomsection\label{\detokenize{bcmdModel:bayescmd.bcmdModel.bcmd_model.ModelBCMD.deleteWorkdir}}\pysigline{\sphinxbfcode{deleteWorkdir}}
\sphinxcode{boolean} \textendash{} Indicates if the working directory should be deleted after finishing.
Default is False, but this will always be set to True if {\hyperref[\detokenize{bcmdModel:bayescmd.bcmdModel.bcmd_model.ModelBCMD.workdir}]{\sphinxcrossref{\sphinxcode{workdir}}}}
is set to None.

\end{fulllineitems}

\index{timeout (bayescmd.bcmdModel.bcmd\_model.ModelBCMD attribute)}

\begin{fulllineitems}
\phantomsection\label{\detokenize{bcmdModel:bayescmd.bcmdModel.bcmd_model.ModelBCMD.timeout}}\pysigline{\sphinxbfcode{timeout}}
\sphinxcode{float} or \sphinxcode{int} \textendash{} Maximum length in seconds to let model run before cancelling. Default
is {\hyperref[\detokenize{bcmdModel:bayescmd.bcmdModel.bcmd_model.TIMEOUT}]{\sphinxcrossref{\sphinxcode{TIMEOUT}}}}.

\end{fulllineitems}

\index{basedir (bayescmd.bcmdModel.bcmd\_model.ModelBCMD attribute)}

\begin{fulllineitems}
\phantomsection\label{\detokenize{bcmdModel:bayescmd.bcmdModel.bcmd_model.ModelBCMD.basedir}}\pysigline{\sphinxbfcode{basedir}}
\sphinxcode{str} \textendash{} Path to base ‘BayesCMD’ directory. By default it is set to
{\hyperref[\detokenize{bcmdModel:bayescmd.bcmdModel.bcmd_model.BASEDIR}]{\sphinxcrossref{\sphinxcode{BASEDIR}}}}.

\end{fulllineitems}

\index{debug (bayescmd.bcmdModel.bcmd\_model.ModelBCMD attribute)}

\begin{fulllineitems}
\phantomsection\label{\detokenize{bcmdModel:bayescmd.bcmdModel.bcmd_model.ModelBCMD.debug}}\pysigline{\sphinxbfcode{debug}}
\sphinxcode{boolean} \textendash{} Indicates if debugging information should be written to console.

\end{fulllineitems}

\index{program (bayescmd.bcmdModel.bcmd\_model.ModelBCMD attribute)}

\begin{fulllineitems}
\phantomsection\label{\detokenize{bcmdModel:bayescmd.bcmdModel.bcmd_model.ModelBCMD.program}}\pysigline{\sphinxbfcode{program}}
\sphinxcode{str} \textendash{} Path to the compiled model file. This is expected to be in
‘{\hyperref[\detokenize{bcmdModel:bayescmd.bcmdModel.bcmd_model.ModelBCMD.basedir}]{\sphinxcrossref{\sphinxcode{basedir}}}}/build’, with the name {\hyperref[\detokenize{bcmdModel:bayescmd.bcmdModel.bcmd_model.ModelBCMD.model_name}]{\sphinxcrossref{\sphinxcode{model\_name}}}}.model.

\end{fulllineitems}



\begin{fulllineitems}
\pysigline{\sphinxbfcode{output\_\{coarse,detail\}}}
\sphinxcode{str} \textendash{} Location to write coarse and detailed output files to. This will be
the working directory, with coarse output files having the suffix
‘.out’ and detailed output files having the suffix ‘.detail’.

\end{fulllineitems}

\index{output\_dict (bayescmd.bcmdModel.bcmd\_model.ModelBCMD attribute)}

\begin{fulllineitems}
\phantomsection\label{\detokenize{bcmdModel:bayescmd.bcmdModel.bcmd_model.ModelBCMD.output_dict}}\pysigline{\sphinxbfcode{output\_dict}}
\sphinxcode{collections.defaultdict(:obj:{}`list}){}` \textendash{} Dictionary of output data.

\end{fulllineitems}

\index{\_cleanupTemp() (bayescmd.bcmdModel.ModelBCMD method)}

\begin{fulllineitems}
\phantomsection\label{\detokenize{bcmdModel:bayescmd.bcmdModel.ModelBCMD._cleanupTemp}}\pysiglinewithargsret{\sphinxbfcode{\_cleanupTemp}}{}{}
Delete working directory.

\end{fulllineitems}

\index{create\_default\_input() (bayescmd.bcmdModel.ModelBCMD method)}

\begin{fulllineitems}
\phantomsection\label{\detokenize{bcmdModel:bayescmd.bcmdModel.ModelBCMD.create_default_input}}\pysiglinewithargsret{\sphinxbfcode{create\_default\_input}}{}{}
Create configured default input file and write to string buffer.

Using this method allows the configured input file to be written to
memory, thus reducing the number of operations involving writing to
disk.

\end{fulllineitems}

\index{create\_initialised\_input() (bayescmd.bcmdModel.ModelBCMD method)}

\begin{fulllineitems}
\phantomsection\label{\detokenize{bcmdModel:bayescmd.bcmdModel.ModelBCMD.create_initialised_input}}\pysiglinewithargsret{\sphinxbfcode{create\_initialised\_input}}{}{}
Create a custom, initialised input and write to buffer.

Create an input file using configuration values specified by
{\hyperref[\detokenize{bcmdModel:bayescmd.bcmdModel.bcmd_model.ModelBCMD.inputs}]{\sphinxcrossref{\sphinxcode{inputs}}}}, {\hyperref[\detokenize{bcmdModel:bayescmd.bcmdModel.bcmd_model.ModelBCMD.times}]{\sphinxcrossref{\sphinxcode{times}}}}, {\hyperref[\detokenize{bcmdModel:bayescmd.bcmdModel.bcmd_model.ModelBCMD.params}]{\sphinxcrossref{\sphinxcode{params}}}} and {\hyperref[\detokenize{bcmdModel:bayescmd.bcmdModel.bcmd_model.ModelBCMD.outputs}]{\sphinxcrossref{\sphinxcode{outputs}}}}, and
write it to the file specified by {\hyperref[\detokenize{bcmdModel:bayescmd.bcmdModel.bcmd_model.ModelBCMD.input_file}]{\sphinxcrossref{\sphinxcode{input\_file}}}}.

The configured input file will be written to memory, thus reducing the
number of operations involving writing to disk.

\end{fulllineitems}

\index{get\_defaults() (bayescmd.bcmdModel.ModelBCMD method)}

\begin{fulllineitems}
\phantomsection\label{\detokenize{bcmdModel:bayescmd.bcmdModel.ModelBCMD.get_defaults}}\pysiglinewithargsret{\sphinxbfcode{get\_defaults}}{}{}
Obtain default model configuration.

\end{fulllineitems}

\index{output\_parse() (bayescmd.bcmdModel.ModelBCMD method)}

\begin{fulllineitems}
\phantomsection\label{\detokenize{bcmdModel:bayescmd.bcmdModel.ModelBCMD.output_parse}}\pysiglinewithargsret{\sphinxbfcode{output\_parse}}{}{}
Parse the output files into a dictionary.

This allows the model output to be sent to JSON, if using WeBCMD, or
simply processed in a more pythonic fashion for any Bayesian (or
similar) analysis.

\end{fulllineitems}

\index{run\_from\_buffer() (bayescmd.bcmdModel.ModelBCMD method)}

\begin{fulllineitems}
\phantomsection\label{\detokenize{bcmdModel:bayescmd.bcmdModel.ModelBCMD.run_from_buffer}}\pysiglinewithargsret{\sphinxbfcode{run\_from\_buffer}}{}{}
Run the model using an input file written to memory.

The input file will need to have been created using
\sphinxcode{create\_initialised\_input} or \sphinxcode{create\_default\_input}.

\end{fulllineitems}

\index{run\_from\_file() (bayescmd.bcmdModel.ModelBCMD method)}

\begin{fulllineitems}
\phantomsection\label{\detokenize{bcmdModel:bayescmd.bcmdModel.ModelBCMD.run_from_file}}\pysiglinewithargsret{\sphinxbfcode{run\_from\_file}}{}{}
Run model using an input file found at {\hyperref[\detokenize{bcmdModel:bayescmd.bcmdModel.bcmd_model.ModelBCMD.input_file}]{\sphinxcrossref{\sphinxcode{input\_file}}}}.

The model will be run using an already created input file that has been
written manually or using the \sphinxcode{write\_default\_input} or
\sphinxcode{write\_initialised\_input} methods.

\end{fulllineitems}

\index{write\_default\_input() (bayescmd.bcmdModel.ModelBCMD method)}

\begin{fulllineitems}
\phantomsection\label{\detokenize{bcmdModel:bayescmd.bcmdModel.ModelBCMD.write_default_input}}\pysiglinewithargsret{\sphinxbfcode{write\_default\_input}}{}{}
Create and write a default input to file.

Create an input file using default configuration values and write it to
the file specified by {\hyperref[\detokenize{bcmdModel:bayescmd.bcmdModel.bcmd_model.ModelBCMD.input_file}]{\sphinxcrossref{\sphinxcode{input\_file}}}}. All inputs, outputs and
parameters are set to default values and there is no burn in.

\end{fulllineitems}

\index{write\_initialised\_input() (bayescmd.bcmdModel.ModelBCMD method)}

\begin{fulllineitems}
\phantomsection\label{\detokenize{bcmdModel:bayescmd.bcmdModel.ModelBCMD.write_initialised_input}}\pysiglinewithargsret{\sphinxbfcode{write\_initialised\_input}}{}{}
Create and write a custom, initialised input to file.

Create an input file using configuration values specified by
{\hyperref[\detokenize{bcmdModel:bayescmd.bcmdModel.bcmd_model.ModelBCMD.inputs}]{\sphinxcrossref{\sphinxcode{inputs}}}}, {\hyperref[\detokenize{bcmdModel:bayescmd.bcmdModel.bcmd_model.ModelBCMD.times}]{\sphinxcrossref{\sphinxcode{times}}}}, {\hyperref[\detokenize{bcmdModel:bayescmd.bcmdModel.bcmd_model.ModelBCMD.params}]{\sphinxcrossref{\sphinxcode{params}}}} and {\hyperref[\detokenize{bcmdModel:bayescmd.bcmdModel.bcmd_model.ModelBCMD.outputs}]{\sphinxcrossref{\sphinxcode{outputs}}}}, and
write it to the file specified by {\hyperref[\detokenize{bcmdModel:bayescmd.bcmdModel.bcmd_model.ModelBCMD.input_file}]{\sphinxcrossref{\sphinxcode{input\_file}}}}.

\end{fulllineitems}


\end{fulllineitems}



\section{Input Creation}
\label{\detokenize{bcmdModel:input-creation}}
Input files are required by the BCMD model. A special class has been created
that will create a correctly formatted input file for a variety of use cases.


\subsection{input\_creation}
\label{\detokenize{bcmdModel:module-bayescmd.bcmdModel.input_creation}}\label{\detokenize{bcmdModel:id1}}\index{bayescmd.bcmdModel.input\_creation (module)}
Create input files for use with a BCMD model.

Input files are needed in order to set model parameters and provide driving
inputs.
\index{InputCreator (class in bayescmd.bcmdModel)}

\begin{fulllineitems}
\phantomsection\label{\detokenize{bcmdModel:bayescmd.bcmdModel.InputCreator}}\pysiglinewithargsret{\sphinxbfcode{class }\sphinxcode{bayescmd.bcmdModel.}\sphinxbfcode{InputCreator}}{\emph{times}, \emph{inputs}, \emph{outputs=None}, \emph{params=None}, \emph{filename=None}}{}
Create an input file by passing relevant information to the class.

This input file is then used to create an input file that can either be
written to file or kept in buffer and passed directly to the model.
\begin{quote}\begin{description}
\item[{Parameters}] \leavevmode\begin{itemize}
\item {} 
\sphinxstyleliteralstrong{times} (\sphinxcode{list} of \sphinxcode{float} or \sphinxcode{int}) \textendash{} List of times at which measurement data has been collected and needs
to be simulated.

\item {} 
\sphinxstyleliteralstrong{inputs} (\sphinxstyleliteralemphasis{dict}) \textendash{} Dictionary of model inputs and their values. Has form
\{‘names’ : \sphinxcode{list} of \sphinxcode{str},
‘values’ : \sphinxcode{list} of \sphinxcode{list} of \sphinxcode{float}\}
where \sphinxtitleref{names} should be a list of each model input name, matching up to
the model inputs and \sphinxtitleref{values} would be a list of lists, where each
sublist is the input values for that time point. With this in mind,
the length of \sphinxtitleref{inputs{[}‘values’{]}{}`} should equal length of \sphinxtitleref{times}.

\item {} 
\sphinxstyleliteralstrong{filename} (\sphinxcode{str}, optional) \textendash{} Name of the input file to be written to if writing to file is required.
Default is \sphinxcode{None}.

\item {} 
\sphinxstyleliteralstrong{params} (\sphinxcode{dict} of \sphinxcode{str}: \sphinxcode{float}, optional) \textendash{} Dictionary of \{‘parameter’: param\_value\}

\item {} 
\sphinxstyleliteralstrong{outputs} (\sphinxcode{list} of \sphinxcode{str}, optional) \textendash{} List of model outputs to return.

\end{itemize}

\end{description}\end{quote}
\index{times (bayescmd.bcmdModel.input\_creation.InputCreator attribute)}

\begin{fulllineitems}
\phantomsection\label{\detokenize{bcmdModel:bayescmd.bcmdModel.input_creation.InputCreator.times}}\pysigline{\sphinxbfcode{times}}
\sphinxcode{list} of \sphinxcode{float} or \sphinxcode{int} \textendash{} List of times at which measurement data has been collected and needs
to be simulated.

\end{fulllineitems}

\index{inputs (bayescmd.bcmdModel.input\_creation.InputCreator attribute)}

\begin{fulllineitems}
\phantomsection\label{\detokenize{bcmdModel:bayescmd.bcmdModel.input_creation.InputCreator.inputs}}\pysigline{\sphinxbfcode{inputs}}
\sphinxstyleemphasis{dict} \textendash{} Dictionary of model inputs and their values. Has form
\{‘names’ : \sphinxcode{list} of \sphinxcode{str},
‘values’ : \sphinxcode{list} of \sphinxcode{list} of \sphinxcode{float}\}
where \sphinxtitleref{names} should be a list of each model input name, matching up to
the model inputs and \sphinxtitleref{values} would be a list of lsits, where each
sublist is the input values for that time point. With this in mind,
the length of \sphinxtitleref{inputs{[}‘values’{]}{}`} should equal length of \sphinxtitleref{times}.

\end{fulllineitems}

\index{f\_out (bayescmd.bcmdModel.input\_creation.InputCreator attribute)}

\begin{fulllineitems}
\phantomsection\label{\detokenize{bcmdModel:bayescmd.bcmdModel.input_creation.InputCreator.f_out}}\pysigline{\sphinxbfcode{f\_out}}
\sphinxcode{StringIO()} \textendash{} String buffer object to which the input file will be written.

\end{fulllineitems}

\index{filename (bayescmd.bcmdModel.input\_creation.InputCreator attribute)}

\begin{fulllineitems}
\phantomsection\label{\detokenize{bcmdModel:bayescmd.bcmdModel.input_creation.InputCreator.filename}}\pysigline{\sphinxbfcode{filename}}
\sphinxstyleemphasis{str} \textendash{} Name of the input file to be written to if writing to file is required.
Default is \sphinxcode{None}.

\end{fulllineitems}

\index{params (bayescmd.bcmdModel.input\_creation.InputCreator attribute)}

\begin{fulllineitems}
\phantomsection\label{\detokenize{bcmdModel:bayescmd.bcmdModel.input_creation.InputCreator.params}}\pysigline{\sphinxbfcode{params}}
dict of \sphinxcode{str}: \sphinxcode{float}. \textendash{} Dictionary of \{‘parameter’: param\_value\}

\end{fulllineitems}

\index{outputs (bayescmd.bcmdModel.input\_creation.InputCreator attribute)}

\begin{fulllineitems}
\phantomsection\label{\detokenize{bcmdModel:bayescmd.bcmdModel.input_creation.InputCreator.outputs}}\pysigline{\sphinxbfcode{outputs}}
\sphinxcode{list} of \sphinxcode{str} \textendash{} List of model outputs to return.

\end{fulllineitems}

\index{default\_creation() (bayescmd.bcmdModel.InputCreator method)}

\begin{fulllineitems}
\phantomsection\label{\detokenize{bcmdModel:bayescmd.bcmdModel.InputCreator.default_creation}}\pysiglinewithargsret{\sphinxbfcode{default\_creation}}{}{}
Create a default input file from given arguments.

Assumes parameters remain unchanged from default values.
\begin{quote}\begin{description}
\item[{Returns}] \leavevmode
Returns the input file as a String.IO() buffer object.

\item[{Return type}] \leavevmode
\sphinxcode{String.IO()}

\end{description}\end{quote}

\end{fulllineitems}

\index{initialised\_creation() (bayescmd.bcmdModel.InputCreator method)}

\begin{fulllineitems}
\phantomsection\label{\detokenize{bcmdModel:bayescmd.bcmdModel.InputCreator.initialised_creation}}\pysiglinewithargsret{\sphinxbfcode{initialised\_creation}}{\emph{burn\_in}}{}
Create an input file from given arguments.

Creates an input file thatcan have non-default parameter values and
outputs, as well as a burn in period. Assumes parameters remain
constant for the full duration of the simulation.
\begin{quote}\begin{description}
\item[{Parameters}] \leavevmode
\sphinxstyleliteralstrong{burn\_in} (\sphinxcode{float} or \sphinxcode{int}) \textendash{} Length of burn in period at start of the simulation.

\item[{Returns}] \leavevmode
Returns the input file as a String.IO() buffer object.

\item[{Return type}] \leavevmode
\sphinxcode{String.IO()}

\end{description}\end{quote}

\end{fulllineitems}

\index{input\_file\_write() (bayescmd.bcmdModel.InputCreator method)}

\begin{fulllineitems}
\phantomsection\label{\detokenize{bcmdModel:bayescmd.bcmdModel.InputCreator.input_file_write}}\pysiglinewithargsret{\sphinxbfcode{input\_file\_write}}{}{}
Write input file from buffer to file.

\end{fulllineitems}


\end{fulllineitems}



\chapter{abc}
\label{\detokenize{abc::doc}}\label{\detokenize{abc:abc}}
The \sphinxtitleref{abc} subpackage is used to handle the Approximate Bayesian Computation
(ABC) specific components of BayesCMD. This includes running the model multiple
times in a batch process, calculating distances between datasets and generating
priors for parameters.
\phantomsection\label{\detokenize{abc:module-bayescmd.abc}}\index{bayescmd.abc (module)}

\section{Distances}
\label{\detokenize{abc:module-bayescmd.abc.distances}}\label{\detokenize{abc:distances}}\index{bayescmd.abc.distances (module)}
Use to generate distance measures between simulated and real time series.
\index{DISTANCES (in module bayescmd.abc.distances)}

\begin{fulllineitems}
\phantomsection\label{\detokenize{abc:bayescmd.abc.distances.DISTANCES}}\pysigline{\sphinxcode{bayescmd.abc.distances.}\sphinxbfcode{DISTANCES}}
\sphinxstyleemphasis{dict} \textendash{} Dictionary contianing the distance aliases, mapping to the functions.

\end{fulllineitems}

\index{Error}

\begin{fulllineitems}
\phantomsection\label{\detokenize{abc:bayescmd.abc.distances.Error}}\pysigline{\sphinxbfcode{exception }\sphinxcode{bayescmd.abc.distances.}\sphinxbfcode{Error}}
Base class for exceptions in this module.

\end{fulllineitems}

\index{ZeroArrayError}

\begin{fulllineitems}
\phantomsection\label{\detokenize{abc:bayescmd.abc.distances.ZeroArrayError}}\pysigline{\sphinxbfcode{exception }\sphinxcode{bayescmd.abc.distances.}\sphinxbfcode{ZeroArrayError}}
Exception raised for errors in the zero array.

\end{fulllineitems}

\index{check\_for\_key() (in module bayescmd.abc.distances)}

\begin{fulllineitems}
\phantomsection\label{\detokenize{abc:bayescmd.abc.distances.check_for_key}}\pysiglinewithargsret{\sphinxcode{bayescmd.abc.distances.}\sphinxbfcode{check\_for\_key}}{\emph{dictionary}, \emph{target}}{}
Check that a dictionary contains a key, and if so, return its data.
\begin{quote}\begin{description}
\item[{Parameters}] \leavevmode\begin{itemize}
\item {} 
\sphinxstyleliteralstrong{dictionary} (\sphinxstyleliteralemphasis{dict}) \textendash{} Dictionary to check for \sphinxtitleref{target} key.

\item {} 
\sphinxstyleliteralstrong{target} (\sphinxstyleliteralemphasis{str}) \textendash{} String containing the target variable that is expected to be found in
\sphinxtitleref{dictionary}

\end{itemize}

\item[{Returns}] \leavevmode
\sphinxstylestrong{data} \textendash{} List of data found in \sphinxtitleref{dictionary}. This is likely to be the time
series data collected experimentally or generated by the model.

\item[{Return type}] \leavevmode
list

\end{description}\end{quote}

\end{fulllineitems}

\index{euclidean\_dist() (in module bayescmd.abc.distances)}

\begin{fulllineitems}
\phantomsection\label{\detokenize{abc:bayescmd.abc.distances.euclidean_dist}}\pysiglinewithargsret{\sphinxcode{bayescmd.abc.distances.}\sphinxbfcode{euclidean\_dist}}{\emph{data1}, \emph{data2}}{}
Get the euclidean distance between two numpy arrays.
\begin{quote}\begin{description}
\item[{Parameters}] \leavevmode\begin{itemize}
\item {} 
\sphinxstyleliteralstrong{data1} (\sphinxstyleliteralemphasis{np.ndarray}) \textendash{} 
First data array.

The shape should match that of data2 and the number of rows should
match the number of model outputs i.e. 2 model outputs will be two
rows.


\item {} 
\sphinxstyleliteralstrong{data2} (\sphinxstyleliteralemphasis{np.ndarray}) \textendash{} 
Second data array.

The shape should match that of data1 and the number of rows should
match the number of model outputs i.e. 2 model outputs will be two
rows.


\end{itemize}

\item[{Returns}] \leavevmode
\sphinxstylestrong{d} \textendash{} Euclidean distance measure

\item[{Return type}] \leavevmode
float

\end{description}\end{quote}

\end{fulllineitems}

\index{get\_distance() (in module bayescmd.abc.distances)}

\begin{fulllineitems}
\phantomsection\label{\detokenize{abc:bayescmd.abc.distances.get_distance}}\pysiglinewithargsret{\sphinxcode{bayescmd.abc.distances.}\sphinxbfcode{get\_distance}}{\emph{actual\_data}, \emph{sim\_data}, \emph{targets}, \emph{zero\_flag}, \emph{distance='euclidean'}, \emph{normalise=False}}{}
Obtain  distance between two sets of data.

Get a distance as defined by \sphinxtitleref{distance} between two sets of data as well
as between each signal in the data.
\begin{quote}\begin{description}
\item[{Parameters}] \leavevmode\begin{itemize}
\item {} 
\sphinxstyleliteralstrong{actual\_data} (\sphinxstyleliteralemphasis{dict}) \textendash{} Dictionary of actual data, as generated by
\sphinxcode{bayescmd.abc.data\_import.import\_actual\_data()}

\item {} 
\sphinxstyleliteralstrong{sim\_data} (\sphinxstyleliteralemphasis{dict}) \textendash{} Dictionary of simulated data, as created by
{\hyperref[\detokenize{bcmdModel:bayescmd.bcmdModel.ModelBCMD.output_parse}]{\sphinxcrossref{\sphinxcode{bayescmd.bcmdModel.ModelBCMD.output\_parse()}}}}

\item {} 
\sphinxstyleliteralstrong{targets} (list of \sphinxcode{str}) \textendash{} List of model targets, which should all be strings.

\item {} 
\sphinxstyleliteralstrong{zero\_flag} (\sphinxstyleliteralemphasis{dict}) \textendash{} 
Dictionary of form target(\sphinxcode{str}): bool, where bool indicates
whether to zero that target.

Note: zero\_flag keys should match targets list.


\item {} 
\sphinxstyleliteralstrong{distance} (\sphinxstyleliteralemphasis{str}\sphinxstyleliteralemphasis{, }\sphinxstyleliteralemphasis{optional}) \textendash{} Name of distance measure to use. One of {[}‘euclidean’, ‘manhattan’,
‘MAE’, ‘MSE’{]}, where default is ‘euclidean’.

\item {} 
\sphinxstyleliteralstrong{normalise} (\sphinxstyleliteralemphasis{bool}\sphinxstyleliteralemphasis{, }\sphinxstyleliteralemphasis{optional}) \textendash{} Boolean flag to indicate whether the signals need normalising, default
is False. Current normalisation is done using z-score but that is
likely to change with time.

\end{itemize}

\item[{Returns}] \leavevmode

\sphinxstylestrong{distances} \textendash{}
\begin{description}
\item[{Dictionary of form:}] \leavevmode
\{‘TOTAL’: summed distance of all signals,
‘target1: distance of 1st target’,
…
‘targetN’: distance of Nth target
\}

\end{description}


\item[{Return type}] \leavevmode
dict

\end{description}\end{quote}

\end{fulllineitems}

\index{manhattan\_dist() (in module bayescmd.abc.distances)}

\begin{fulllineitems}
\phantomsection\label{\detokenize{abc:bayescmd.abc.distances.manhattan_dist}}\pysiglinewithargsret{\sphinxcode{bayescmd.abc.distances.}\sphinxbfcode{manhattan\_dist}}{\emph{data1}, \emph{data2}}{}
Get the Manhattan distance between two numpy arrays.
\begin{quote}\begin{description}
\item[{Parameters}] \leavevmode\begin{itemize}
\item {} 
\sphinxstyleliteralstrong{data1} (\sphinxstyleliteralemphasis{np.ndarray}) \textendash{} 
First data array.

The shape should match that of data2 and the number of rows should
match the number of model outputs i.e. 2 model outputs will be two
rows.


\item {} 
\sphinxstyleliteralstrong{data2} (\sphinxstyleliteralemphasis{np.ndarray}) \textendash{} 
Second data array.

The shape should match that of data1 and the number of rows should
match the number of model outputs i.e. 2 model outputs will be two
rows.


\end{itemize}

\item[{Returns}] \leavevmode
\sphinxstylestrong{d} \textendash{} Manhattan distance measure

\item[{Return type}] \leavevmode
float

\end{description}\end{quote}

\end{fulllineitems}

\index{mean\_absolute\_error\_dist() (in module bayescmd.abc.distances)}

\begin{fulllineitems}
\phantomsection\label{\detokenize{abc:bayescmd.abc.distances.mean_absolute_error_dist}}\pysiglinewithargsret{\sphinxcode{bayescmd.abc.distances.}\sphinxbfcode{mean\_absolute\_error\_dist}}{\emph{data1}, \emph{data2}}{}
Get the normalised manhattan distance between two numpy arrays.
\begin{quote}\begin{description}
\item[{Parameters}] \leavevmode\begin{itemize}
\item {} 
\sphinxstyleliteralstrong{data1} (\sphinxstyleliteralemphasis{np.ndarray}) \textendash{} 
First data array.

The shape should match that of data2 and the number of rows should
match the number of model outputs i.e. 2 model outputs will be two
rows.


\item {} 
\sphinxstyleliteralstrong{data2} (\sphinxstyleliteralemphasis{np.ndarray}) \textendash{} 
Second data array.

The shape should match that of data1 and the number of rows should
match the number of model outputs i.e. 2 model outputs will be two
rows.


\end{itemize}

\item[{Returns}] \leavevmode
\sphinxstylestrong{d} \textendash{} Normalised Manhattan distance measure

\item[{Return type}] \leavevmode
float

\end{description}\end{quote}

\end{fulllineitems}

\index{mean\_square\_error\_dist() (in module bayescmd.abc.distances)}

\begin{fulllineitems}
\phantomsection\label{\detokenize{abc:bayescmd.abc.distances.mean_square_error_dist}}\pysiglinewithargsret{\sphinxcode{bayescmd.abc.distances.}\sphinxbfcode{mean\_square\_error\_dist}}{\emph{data1}, \emph{data2}}{}
Get the Mean Square Error distance between two numpy arrays.
\begin{quote}\begin{description}
\item[{Parameters}] \leavevmode\begin{itemize}
\item {} 
\sphinxstyleliteralstrong{data1} (\sphinxstyleliteralemphasis{np.ndarray}) \textendash{} 
First data array.

The shape should match that of data2 and the number of rows should
match the number of model outputs i.e. 2 model outputs will be two
rows.


\item {} 
\sphinxstyleliteralstrong{data2} (\sphinxstyleliteralemphasis{np.ndarray}) \textendash{} 
Second data array.

The shape should match that of data1 and the number of rows should
match the number of model outputs i.e. 2 model outputs will be two
rows.


\end{itemize}

\item[{Returns}] \leavevmode
\sphinxstylestrong{d} \textendash{} Mean Square Error distance measure

\item[{Return type}] \leavevmode
float

\end{description}\end{quote}

\end{fulllineitems}

\index{zero\_array() (in module bayescmd.abc.distances)}

\begin{fulllineitems}
\phantomsection\label{\detokenize{abc:bayescmd.abc.distances.zero_array}}\pysiglinewithargsret{\sphinxcode{bayescmd.abc.distances.}\sphinxbfcode{zero\_array}}{\emph{array}, \emph{zero\_flag}}{}
Zero an array of data with its initial values.
\begin{quote}\begin{description}
\item[{Parameters}] \leavevmode\begin{itemize}
\item {} 
\sphinxstyleliteralstrong{array} (\sphinxstyleliteralemphasis{list}) \textendash{} List of data

\item {} 
\sphinxstyleliteralstrong{zero\_flags} (\sphinxstyleliteralemphasis{bool}) \textendash{} Boolean indicating if data needs zeroing

\end{itemize}

\item[{Returns}] \leavevmode
\sphinxstylestrong{zerod} \textendash{} Zero’d list

\item[{Return type}] \leavevmode
list

\end{description}\end{quote}

\end{fulllineitems}



\chapter{jsonParsing}
\label{\detokenize{jsonParsing:module-bayescmd.jsonParsing.modelJSON}}\label{\detokenize{jsonParsing:jsonparsing}}\label{\detokenize{jsonParsing::doc}}\index{bayescmd.jsonParsing.modelJSON (module)}
Convert model information to JSON for use with WeBCMD.
\index{float\_or\_str() (in module bayescmd.jsonParsing.modelJSON)}

\begin{fulllineitems}
\phantomsection\label{\detokenize{jsonParsing:bayescmd.jsonParsing.modelJSON.float_or_str}}\pysiglinewithargsret{\sphinxcode{bayescmd.jsonParsing.modelJSON.}\sphinxbfcode{float\_or\_str}}{\emph{n}}{}
Determine if a string can be returned as a float.
\begin{quote}\begin{description}
\item[{Parameters}] \leavevmode
\sphinxstyleliteralstrong{n} (\sphinxcode{str}) \textendash{} String to convert to float if possible

\item[{Returns}] \leavevmode
\sphinxstylestrong{s} \textendash{} \sphinxtitleref{n} as \sphinxcode{float}, or \sphinxcode{str} if not a number.

\item[{Return type}] \leavevmode
\sphinxcode{float}

\end{description}\end{quote}

\end{fulllineitems}

\index{getDefaultFilePath() (in module bayescmd.jsonParsing.modelJSON)}

\begin{fulllineitems}
\phantomsection\label{\detokenize{jsonParsing:bayescmd.jsonParsing.modelJSON.getDefaultFilePath}}\pysiglinewithargsret{\sphinxcode{bayescmd.jsonParsing.modelJSON.}\sphinxbfcode{getDefaultFilePath}}{\emph{model\_name}}{}
Given a model name, return the default path to the modeldef file.
\begin{quote}\begin{description}
\item[{Parameters}] \leavevmode
\sphinxstyleliteralstrong{model\_name} (\sphinxstyleliteralemphasis{str}) \textendash{} Name of model

\item[{Returns}] \leavevmode
\sphinxstylestrong{modelPath} \textendash{} Path to modeldef file

\item[{Return type}] \leavevmode
str

\end{description}\end{quote}

\end{fulllineitems}

\index{get\_model\_name() (in module bayescmd.jsonParsing.modelJSON)}

\begin{fulllineitems}
\phantomsection\label{\detokenize{jsonParsing:bayescmd.jsonParsing.modelJSON.get_model_name}}\pysiglinewithargsret{\sphinxcode{bayescmd.jsonParsing.modelJSON.}\sphinxbfcode{get\_model\_name}}{\emph{fpath}}{}
Return model name from file path.
\begin{quote}\begin{description}
\item[{Parameters}] \leavevmode
\sphinxstyleliteralstrong{fpath} (\sphinxcode{str}) \textendash{} Path to model def file.

\item[{Returns}] \leavevmode
Name of model, as determined by modeldef file.

\item[{Return type}] \leavevmode
\sphinxcode{str}

\end{description}\end{quote}

\end{fulllineitems}

\index{json\_writer() (in module bayescmd.jsonParsing.modelJSON)}

\begin{fulllineitems}
\phantomsection\label{\detokenize{jsonParsing:bayescmd.jsonParsing.modelJSON.json_writer}}\pysiglinewithargsret{\sphinxcode{bayescmd.jsonParsing.modelJSON.}\sphinxbfcode{json\_writer}}{\emph{model\_name}, \emph{dictionary}}{}
Write a python dictionary to JSON.
\begin{quote}\begin{description}
\item[{Parameters}] \leavevmode\begin{itemize}
\item {} 
\sphinxstyleliteralstrong{model\_name} (\sphinxcode{str}) \textendash{} Name of BCMD model.

\item {} 
\sphinxstyleliteralstrong{dictionary} (\sphinxcode{dict}) \textendash{} Dictionary to write to file.

\end{itemize}

\item[{Returns}] \leavevmode
Writes to JSON file in \sphinxcode{data/} dir with name \sphinxtitleref{model\_name}.json

\item[{Return type}] \leavevmode
None

\end{description}\end{quote}

\end{fulllineitems}

\index{modeldefParse() (in module bayescmd.jsonParsing.modelJSON)}

\begin{fulllineitems}
\phantomsection\label{\detokenize{jsonParsing:bayescmd.jsonParsing.modelJSON.modeldefParse}}\pysiglinewithargsret{\sphinxcode{bayescmd.jsonParsing.modelJSON.}\sphinxbfcode{modeldefParse}}{\emph{fpath}}{}
Process a modeldef file to extract information.

Function reads a modeldef file and extracts default model inputs, outputs
and parameters.
\begin{quote}\begin{description}
\item[{Parameters}] \leavevmode
\sphinxstyleliteralstrong{fpath} (\sphinxcode{str}) \textendash{} Path to modeldef file.

\item[{Returns}] \leavevmode

\sphinxstylestrong{model\_data} \textendash{} Dictionary of model information with form:

\{model\_name: name of model, obtained via {\hyperref[\detokenize{jsonParsing:bayescmd.jsonParsing.modelJSON.get_model_name}]{\sphinxcrossref{\sphinxcode{get\_model\_name()}}}},

input: \sphinxcode{list} of \sphinxcode{str} for each model input,

output: \sphinxcode{list} of \sphinxcode{str} for each model output,

parameters: \sphinxcode{dict} of param\_name (\sphinxcode{str}):
param\_value(\sphinxcode{str})\}


\item[{Return type}] \leavevmode
\sphinxcode{dict}

\end{description}\end{quote}

\end{fulllineitems}



\chapter{Miscellaneous}
\label{\detokenize{misc:miscellaneous}}\label{\detokenize{misc::doc}}
Here you will find a number of useful functions that are used throughout
the general BayesCMD package.


\section{Utility Functions}
\label{\detokenize{misc:utility-functions}}\label{\detokenize{misc:module-bayescmd.util}}\index{bayescmd.util (module)}
Miscellaneous utility functions used throughout BayesCMD.

This module contains a number of utility functions that are used throughout
the different BayesCMD subpackages.
\index{findBaseDir() (in module bayescmd.util)}

\begin{fulllineitems}
\phantomsection\label{\detokenize{misc:bayescmd.util.findBaseDir}}\pysiglinewithargsret{\sphinxcode{bayescmd.util.}\sphinxbfcode{findBaseDir}}{\emph{basename}, \emph{max\_depth=5}, \emph{verbose=False}}{}
Get relative path to a BASEDIR.
:param basename: Name of the basedir to path to
:type basename: str
\begin{quote}\begin{description}
\item[{Returns}] \leavevmode
Relative path to base directory.

\item[{Return type}] \leavevmode
StringIO

\end{description}\end{quote}

\end{fulllineitems}

\index{round\_sig() (in module bayescmd.util)}

\begin{fulllineitems}
\phantomsection\label{\detokenize{misc:bayescmd.util.round_sig}}\pysiglinewithargsret{\sphinxcode{bayescmd.util.}\sphinxbfcode{round\_sig}}{\emph{x}, \emph{sig=1}}{}
Round a value to N sig fig.
\begin{quote}\begin{description}
\item[{Parameters}] \leavevmode\begin{itemize}
\item {} 
\sphinxstyleliteralstrong{x} (\sphinxstyleliteralemphasis{float}) \textendash{} Value to round

\item {} 
\sphinxstyleliteralstrong{sig} (\sphinxstyleliteralemphasis{int}\sphinxstyleliteralemphasis{, }\sphinxstyleliteralemphasis{optional}) \textendash{} Number of sig figs, default is 1

\end{itemize}

\item[{Returns}] \leavevmode
Rounded value

\item[{Return type}] \leavevmode
float

\end{description}\end{quote}

\end{fulllineitems}



\section{Processing Results}
\label{\detokenize{misc:module-bayescmd.results_handling}}\label{\detokenize{misc:processing-results}}\index{bayescmd.results\_handling (module)}
Process results obtained using BayesCMD.

Process the various results obtained using BayesCMD, such as the
\sphinxtitleref{parameters.csv} file. It is also possible to concatenate a number of different
\sphinxtitleref{parameters.csv} files obtained using parallel batch runs into a single
parameters file.
\index{BAYESCMD (in module bayescmd.results\_handling)}

\begin{fulllineitems}
\phantomsection\label{\detokenize{misc:bayescmd.results_handling.BAYESCMD}}\pysigline{\sphinxcode{bayescmd.results\_handling.}\sphinxbfcode{BAYESCMD}}
\sphinxcode{str} \textendash{} Absolute path to base directory. Found using
{\hyperref[\detokenize{misc:bayescmd.util.findBaseDir}]{\sphinxcrossref{\sphinxcode{bayescmd.util.findBaseDir}}}}

\end{fulllineitems}

\index{data\_import() (in module bayescmd.results\_handling)}

\begin{fulllineitems}
\phantomsection\label{\detokenize{misc:bayescmd.results_handling.data_import}}\pysiglinewithargsret{\sphinxcode{bayescmd.results\_handling.}\sphinxbfcode{data\_import}}{\emph{pfile}, \emph{nan\_sub=100000}, \emph{chunk\_size=10000}, \emph{verbose=True}}{}
Import a parameters file produced by a batch process.
\begin{quote}\begin{description}
\item[{Parameters}] \leavevmode\begin{itemize}
\item {} 
\sphinxstyleliteralstrong{pfile} (\sphinxstyleliteralemphasis{str}) \textendash{} Path to the file of parameters and distances

\item {} 
\sphinxstyleliteralstrong{nan\_sub} (\sphinxstyleliteralemphasis{int}\sphinxstyleliteralemphasis{ or }\sphinxstyleliteralemphasis{float}\sphinxstyleliteralemphasis{, }\sphinxstyleliteralemphasis{optional}) \textendash{} Number to substitute for NaN distances/params. Default of 100000

\item {} 
\sphinxstyleliteralstrong{chunk\_size} (\sphinxstyleliteralemphasis{int}\sphinxstyleliteralemphasis{, }\sphinxstyleliteralemphasis{optional}) \textendash{} Size of chunks to load for dataframe. Default of 10000

\item {} 
\sphinxstyleliteralstrong{verbose} (\sphinxstyleliteralemphasis{bool}\sphinxstyleliteralemphasis{, }\sphinxstyleliteralemphasis{optional}) \textendash{} Boolean as to whether include verbose information. Default of True

\end{itemize}

\item[{Returns}] \leavevmode
\sphinxstylestrong{result} \textendash{} Dataframe containing all the parameters and distances, with NaN swapped
for nan\_sub

\item[{Return type}] \leavevmode
pd.DataFrame

\end{description}\end{quote}

\end{fulllineitems}

\index{data\_merge() (in module bayescmd.results\_handling)}

\begin{fulllineitems}
\phantomsection\label{\detokenize{misc:bayescmd.results_handling.data_merge}}\pysiglinewithargsret{\sphinxcode{bayescmd.results\_handling.}\sphinxbfcode{data\_merge}}{\emph{parent\_directory}, \emph{verbose=True}}{}
Merge a set of parameters.csv files into one.
\begin{quote}\begin{description}
\item[{Parameters}] \leavevmode\begin{itemize}
\item {} 
\sphinxstyleliteralstrong{parent\_directory} (\sphinxcode{list} of \sphinxcode{str}) \textendash{} Parent directory to a set of directories each containing model runs and
a parameters.csv file.

\item {} 
\sphinxstyleliteralstrong{verbose} (\sphinxcode{boolean}, optional) \textendash{} Boolean indicator of whether to print extra information.

\end{itemize}

\item[{Returns}] \leavevmode
Concatenated will be written to file in \sphinxtitleref{parent\_directory}

\item[{Return type}] \leavevmode
None

\end{description}\end{quote}

\end{fulllineitems}

\index{diag\_kde\_plot() (in module bayescmd.results\_handling)}

\begin{fulllineitems}
\phantomsection\label{\detokenize{misc:bayescmd.results_handling.diag_kde_plot}}\pysiglinewithargsret{\sphinxcode{bayescmd.results\_handling.}\sphinxbfcode{diag\_kde\_plot}}{\emph{x}, \emph{medians}, \emph{**kws}}{}
Plot univariate KDE and barplot with median of distribution marked on.

Includes median of distribution as a line and as text.
\begin{quote}\begin{description}
\item[{Parameters}] \leavevmode\begin{itemize}
\item {} 
\sphinxstyleliteralstrong{x} (\sphinxstyleliteralemphasis{array-like}) \textendash{} Array-like of data to plot.

\item {} 
\sphinxstyleliteralstrong{medians} (\sphinxcode{dict}) \textendash{} Dictionary of parameter, median pairings.

\item {} 
\sphinxstyleliteralstrong{kws} (\sphinxstyleliteralemphasis{key}\sphinxstyleliteralemphasis{, }\sphinxstyleliteralemphasis{value pairings.}) \textendash{} Other keyword arguments to pass to \sphinxcode{sns.distplot}.

\end{itemize}

\item[{Returns}] \leavevmode
\sphinxstylestrong{ax} \textendash{} AxesSubplot object of univariate KDE and bar plot with median marked
on as well as text.

\item[{Return type}] \leavevmode
\sphinxcode{matplotlib.AxesSubplot}

\end{description}\end{quote}

\end{fulllineitems}

\index{frac\_calculator() (in module bayescmd.results\_handling)}

\begin{fulllineitems}
\phantomsection\label{\detokenize{misc:bayescmd.results_handling.frac_calculator}}\pysiglinewithargsret{\sphinxcode{bayescmd.results\_handling.}\sphinxbfcode{frac\_calculator}}{\emph{df}, \emph{frac}}{}
Calculate the number of lines for a given fraction.
\begin{quote}\begin{description}
\item[{Parameters}] \leavevmode\begin{itemize}
\item {} 
\sphinxstyleliteralstrong{df} (\sphinxstyleliteralemphasis{pd.DataFrame}) \textendash{} Data frame to find fraction of. Normally the output of
{\hyperref[\detokenize{misc:bayescmd.results_handling.data_import}]{\sphinxcrossref{\sphinxcode{data\_import}}}}

\item {} 
\sphinxstyleliteralstrong{frac} (\sphinxstyleliteralemphasis{float}) \textendash{} The fraction of results to consider. Should be given as a percentage
i.e. 1=1\%, 0.1=0.1\%

\end{itemize}

\item[{Returns}] \leavevmode
Number of lines that make up the fraction.

\item[{Return type}] \leavevmode
int

\end{description}\end{quote}

\end{fulllineitems}

\index{get\_output() (in module bayescmd.results\_handling)}

\begin{fulllineitems}
\phantomsection\label{\detokenize{misc:bayescmd.results_handling.get_output}}\pysiglinewithargsret{\sphinxcode{bayescmd.results\_handling.}\sphinxbfcode{get\_output}}{\emph{model\_name}, \emph{p}, \emph{times}, \emph{input\_data}, \emph{d0}, \emph{targets}, \emph{distance='euclidean'}, \emph{zero\_flag=None}}{}
Generate model output and distances.
\begin{quote}\begin{description}
\item[{Parameters}] \leavevmode\begin{itemize}
\item {} 
\sphinxstyleliteralstrong{model\_name} (\sphinxcode{str}) \textendash{} Name of model

\item {} 
\sphinxstyleliteralstrong{p} (\sphinxcode{dict}) \textendash{} Dict of form \{‘parameter’: value\} for which posteriors are being
investigated.

\item {} 
\sphinxstyleliteralstrong{times} (\sphinxcode{list} of \sphinxcode{float}) \textendash{} List of times at which the data was collected.

\item {} 
\sphinxstyleliteralstrong{input\_data} (\sphinxcode{dict}) \textendash{} Dictionary of input data as generated by \sphinxcode{abc.inputParse}.

\item {} 
\sphinxstyleliteralstrong{d0} (\sphinxcode{dict}) \textendash{} Dictionary of real data, as generated by \sphinxcode{abc.import\_actual\_data}.

\item {} 
\sphinxstyleliteralstrong{targets} (\sphinxcode{list} of \sphinxcode{str}) \textendash{} List of model outputs against which the model is being optimised.

\item {} 
\sphinxstyleliteralstrong{distance} (\sphinxcode{str}) \textendash{} Distance measure. One of ‘euclidean’, ‘manhattan’, ‘MAE’, ‘MSE’.

\item {} 
\sphinxstyleliteralstrong{zero\_flag} (\sphinxstyleliteralemphasis{dict}) \textendash{} 
Dictionary of form target(\sphinxcode{str}): bool, where bool indicates
whether to zero that target.

Note: zero\_flag keys should match targets list.


\end{itemize}

\item[{Returns}] \leavevmode
A tuple of (p, model output data).

\item[{Return type}] \leavevmode
\sphinxcode{tuple}

\end{description}\end{quote}

\end{fulllineitems}

\index{histogram\_plot() (in module bayescmd.results\_handling)}

\begin{fulllineitems}
\phantomsection\label{\detokenize{misc:bayescmd.results_handling.histogram_plot}}\pysiglinewithargsret{\sphinxcode{bayescmd.results\_handling.}\sphinxbfcode{histogram\_plot}}{\emph{df}, \emph{distance='euclidean'}, \emph{fraction=1}, \emph{n\_bins=100}}{}
Plot histogram of distance values.

Plot a histogram showing the distribution of distance values for a given
fraction of all distances in the dataframe. Distance values will have been
calculated during the batch process.
\begin{quote}\begin{description}
\item[{Parameters}] \leavevmode\begin{itemize}
\item {} 
\sphinxstyleliteralstrong{df} (\sphinxcode{pandas.DataFrame}) \textendash{} Dataframe of distances and parameters, generated using
{\hyperref[\detokenize{misc:bayescmd.results_handling.data_import}]{\sphinxcrossref{\sphinxcode{data\_import()}}}}

\item {} 
\sphinxstyleliteralstrong{distance} (\sphinxcode{str}, optional) \textendash{} Distance measure. One of ‘euclidean’, ‘manhattan’, ‘MAE’, ‘MSE’.
Default is ‘euclidean’.

\item {} 
\sphinxstyleliteralstrong{fraction} (\sphinxcode{float}, optional) \textendash{} Fraction of all distances to plot. Varies from 0 to 1. Default is 1.

\item {} 
\sphinxstyleliteralstrong{n\_bins} (\sphinxcode{int}, optional) \textendash{} Number of histogram bins. Default is 100.

\end{itemize}

\item[{Returns}] \leavevmode
Matplotlib figure with histogram on.

\item[{Return type}] \leavevmode
matplotlib.figure

\end{description}\end{quote}

\end{fulllineitems}

\index{kde\_plot() (in module bayescmd.results\_handling)}

\begin{fulllineitems}
\phantomsection\label{\detokenize{misc:bayescmd.results_handling.kde_plot}}\pysiglinewithargsret{\sphinxcode{bayescmd.results\_handling.}\sphinxbfcode{kde\_plot}}{\emph{df}, \emph{params}, \emph{frac}, \emph{plot\_param=1}, \emph{n\_ticks=6}, \emph{d='euclidean'}, \emph{verbose=False}}{}
Plot the model parameters pairwide as a KDE.
\begin{quote}\begin{description}
\item[{Parameters}] \leavevmode\begin{itemize}
\item {} 
\sphinxstyleliteralstrong{df} (\sphinxcode{pandas.DataFrame}) \textendash{} Dataframe of distances and parameters, generated using
{\hyperref[\detokenize{misc:bayescmd.results_handling.data_import}]{\sphinxcrossref{\sphinxcode{data\_import()}}}}

\item {} 
\sphinxstyleliteralstrong{params} (\sphinxcode{dict} of \sphinxcode{str}: \sphinxcode{tuple}) \textendash{} Dict of model parameters to compare, with value tuple of the prior max
and min.

\item {} 
\sphinxstyleliteralstrong{frac} (\sphinxcode{float}) \textendash{} Fraction of results to consider. Should be given as a percentage i.e.
1=1\%, 0.1=0.1\%

\item {} 
\sphinxstyleliteralstrong{plot\_param} (\sphinxcode{int}) \textendash{} 
Which group to plot:
\begin{quote}

0: Outside posterior
1: Inside posterior
2: Failed run
\end{quote}


\item {} 
\sphinxstyleliteralstrong{n\_ticks} (\sphinxcode{int}, optional) \textendash{} Number of x-axis ticks. Useful when a large number of parameters are
bring compared, as the axes can become crowded if the number of ticks
is too high.

\item {} 
\sphinxstyleliteralstrong{d} (\sphinxcode{str}, optional) \textendash{} 
Distance measure. One of ‘euclidean’, ‘manhattan’, ‘MAE’, ‘MSE’.
\begin{quote}

Note: Should be given  as a raw string if latex is used i.e.
\sphinxtitleref{r’MAE’}.
\end{quote}


\item {} 
\sphinxstyleliteralstrong{verbose} (\sphinxcode{boolean}, optional) \textendash{} Boolean to indicate verbosity. Default is False.

\end{itemize}

\item[{Returns}] \leavevmode
\sphinxstylestrong{g} \textendash{} Seaborn pairgrid object is returned in case of further formatting.

\item[{Return type}] \leavevmode
\sphinxcode{seaborn.PairGrid}

\end{description}\end{quote}

\end{fulllineitems}

\index{plot\_repeated\_outputs() (in module bayescmd.results\_handling)}

\begin{fulllineitems}
\phantomsection\label{\detokenize{misc:bayescmd.results_handling.plot_repeated_outputs}}\pysiglinewithargsret{\sphinxcode{bayescmd.results\_handling.}\sphinxbfcode{plot\_repeated\_outputs}}{\emph{df}, \emph{model\_name}, \emph{parameters}, \emph{input\_path}, \emph{inputs}, \emph{targets}, \emph{n\_repeats}, \emph{frac}, \emph{zero\_flag}, \emph{openopt\_path=None}, \emph{distance='euclidean'}}{}
Generate model output and distances multiple times.
\begin{quote}\begin{description}
\item[{Parameters}] \leavevmode\begin{itemize}
\item {} 
\sphinxstyleliteralstrong{model\_name} (\sphinxcode{str}) \textendash{} Name of model. Should match the modeldef file for model being generated
i.e. model\_name of ‘model{}`’ should have a modeldef file
‘model1.modeldef’.

\item {} 
\sphinxstyleliteralstrong{parameters} (\sphinxcode{dict} of \sphinxcode{str}: \sphinxcode{tuple}) \textendash{} Dict of model parameters to compare, with value tuple of the prior max
and min.

\item {} 
\sphinxstyleliteralstrong{input\_path} (\sphinxcode{str}) \textendash{} Path to the true data file

\item {} 
\sphinxstyleliteralstrong{inputs} (\sphinxcode{list} of \sphinxcode{str}) \textendash{} List of model inputs.

\item {} 
\sphinxstyleliteralstrong{targets} (\sphinxcode{list} of \sphinxcode{str}) \textendash{} List of model outputs against which the model is being optimised.

\item {} 
\sphinxstyleliteralstrong{n\_repeats} \textendash{} Number of times to generate output data

\item {} 
\sphinxstyleliteralstrong{frac} (\sphinxcode{float}) \textendash{} Fraction of results to consider. Should be given as a percentage i.e.
1=1\%, 0.1=0.1\%

\item {} 
\sphinxstyleliteralstrong{zero\_flag} (\sphinxstyleliteralemphasis{dict}) \textendash{} 
Dictionary of form target(\sphinxcode{str}): bool, where bool indicates
whether to zero that target.

Note: zero\_flag keys should match targets list.


\item {} 
\sphinxstyleliteralstrong{openopt\_path} (\sphinxcode{str} or \sphinxcode{None}) \textendash{} Path to the openopt data file if it exists. Default is None.

\item {} 
\sphinxstyleliteralstrong{distance} (\sphinxcode{str}, optional) \textendash{} Distance measure. One of ‘euclidean’, ‘manhattan’, ‘MAE’, ‘MSE’.

\end{itemize}

\item[{Returns}] \leavevmode
\sphinxstylestrong{fig} \textendash{} Figure containing all axes.

\item[{Return type}] \leavevmode
\sphinxcode{matplotlib.figure}

\end{description}\end{quote}

\end{fulllineitems}

\index{run\_model() (in module bayescmd.results\_handling)}

\begin{fulllineitems}
\phantomsection\label{\detokenize{misc:bayescmd.results_handling.run_model}}\pysiglinewithargsret{\sphinxcode{bayescmd.results\_handling.}\sphinxbfcode{run\_model}}{\emph{model}}{}
Run a BCMD Model.
\begin{quote}\begin{description}
\item[{Parameters}] \leavevmode
\sphinxstyleliteralstrong{model} ({\hyperref[\detokenize{bcmdModel:bayescmd.bcmdModel.ModelBCMD}]{\sphinxcrossref{\sphinxcode{bayescmd.bcmdModel.ModelBCMD}}}}) \textendash{} An initialised instance of a ModelBCMD class.

\item[{Returns}] \leavevmode
\sphinxstylestrong{output} \textendash{} Dictionary of parsed model output.

\item[{Return type}] \leavevmode
\sphinxcode{dict}

\end{description}\end{quote}

\end{fulllineitems}

\index{scatter\_dist\_plot() (in module bayescmd.results\_handling)}

\begin{fulllineitems}
\phantomsection\label{\detokenize{misc:bayescmd.results_handling.scatter_dist_plot}}\pysiglinewithargsret{\sphinxcode{bayescmd.results\_handling.}\sphinxbfcode{scatter\_dist\_plot}}{\emph{df}, \emph{params}, \emph{frac}, \emph{n\_ticks=6}, \emph{d='euclidean'}, \emph{verbose=False}}{}
Plot distribution of parameters as a scatter PairPlot.
\begin{quote}\begin{description}
\item[{Parameters}] \leavevmode\begin{itemize}
\item {} 
\sphinxstyleliteralstrong{df} (\sphinxcode{pandas.DataFrame}) \textendash{} Dataframe of distances and parameters, generated using
{\hyperref[\detokenize{misc:bayescmd.results_handling.data_import}]{\sphinxcrossref{\sphinxcode{data\_import()}}}}

\item {} 
\sphinxstyleliteralstrong{params} (\sphinxcode{dict} of \sphinxcode{str}: \sphinxcode{tuple}) \textendash{} Dict of model parameters to compare, with value tuple of the prior max
and min.

\item {} 
\sphinxstyleliteralstrong{frac} (\sphinxcode{float}) \textendash{} Fraction of results to consider. Should be given as a percentage i.e.
1=1\%, 0.1=0.1\%

\item {} 
\sphinxstyleliteralstrong{n\_ticks} (\sphinxcode{int}, optional) \textendash{} Number of x-axis ticks. Useful when a large number of parameters are
bring compared, as the axes can become crowded if the number of ticks
is too high.

\item {} 
\sphinxstyleliteralstrong{d} (\sphinxcode{str}, optional) \textendash{} 
Distance measure. One of ‘euclidean’, ‘manhattan’, ‘MAE’, ‘MSE’.
\begin{quote}

Note: Should be given  as a raw string if latex is used i.e.
\sphinxtitleref{r’MAE’}.
\end{quote}


\item {} 
\sphinxstyleliteralstrong{verbose} (\sphinxcode{boolean}, optional) \textendash{} Boolean to indicate verbosity. Default is False.

\end{itemize}

\item[{Returns}] \leavevmode
\sphinxstylestrong{g} \textendash{} Seaborn pairgrid object is returned in case of further formatting.

\item[{Return type}] \leavevmode
\sphinxcode{seaborn.PairGrid}

\end{description}\end{quote}

\end{fulllineitems}



\chapter{Indices and tables}
\label{\detokenize{index:indices-and-tables}}\begin{itemize}
\item {} 
\DUrole{xref,std,std-ref}{genindex}

\item {} 
\DUrole{xref,std,std-ref}{modindex}

\item {} 
\DUrole{xref,std,std-ref}{search}

\end{itemize}


\renewcommand{\indexname}{Python Module Index}
\begin{sphinxtheindex}
\def\bigletter#1{{\Large\sffamily#1}\nopagebreak\vspace{1mm}}
\bigletter{b}
\item {\sphinxstyleindexentry{bayescmd.abc}}\sphinxstyleindexpageref{abc:\detokenize{module-bayescmd.abc}}
\item {\sphinxstyleindexentry{bayescmd.abc.distances}}\sphinxstyleindexpageref{abc:\detokenize{module-bayescmd.abc.distances}}
\item {\sphinxstyleindexentry{bayescmd.bcmdModel.bcmd\_model}}\sphinxstyleindexpageref{bcmdModel:\detokenize{module-bayescmd.bcmdModel.bcmd_model}}
\item {\sphinxstyleindexentry{bayescmd.bcmdModel.input\_creation}}\sphinxstyleindexpageref{bcmdModel:\detokenize{module-bayescmd.bcmdModel.input_creation}}
\item {\sphinxstyleindexentry{bayescmd.jsonParsing.modelJSON}}\sphinxstyleindexpageref{jsonParsing:\detokenize{module-bayescmd.jsonParsing.modelJSON}}
\item {\sphinxstyleindexentry{bayescmd.results\_handling}}\sphinxstyleindexpageref{misc:\detokenize{module-bayescmd.results_handling}}
\item {\sphinxstyleindexentry{bayescmd.util}}\sphinxstyleindexpageref{misc:\detokenize{module-bayescmd.util}}
\end{sphinxtheindex}

\renewcommand{\indexname}{Index}
\printindex
\end{document}